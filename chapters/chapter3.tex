\chapter{Requirements / Design Goals}
\label{requirements}

The basic requirement of our application is to ease the comparative study of literatures and give the reader flexibility to focus on the area of his interest. Based on the related studies, cognitive thinking and discussions we derived following requirements for our visual analysis tool for literature comparison.

R1: Provide (abstract) overview
The comparison task includes estimation of the similarities or dissimilarities between two things like books, concepts. For understanding topics and stories in a literature, it is beneficial to show generalised overview of its contents. It gives the reader possibility to compare literatures at a glance. In order to support the comparison tasks, the important similarities and differences in the literature should be emphasised while hiding the less frequent topics and their context.

R2: Provide relevance of the abstracted information
Even if literatures share same topics, the context in which topics are described might be different. Hence it is important to provide the context for such topics. The relevance/ focus of topics in literature is not always same i.e. a topic in a literature is described in depth whereas; other topics are used as references or support. Such relevance is of reader's interest and hence it must be informed to the reader.

R3: Indicate similarity between literature
For aiding the comparison tasks, information on similarity and difference between literatures should be visualised in such a way that a reader can comprehend resemblance easily. This information on topics and their context assist readers to understand how storylines in the literatures are similar and/ or dependent on each other. The visualisation of the difference between literatures rounds up the overview of the interesting topics.

R4: Changing Content
 The proper functionalities should be provided to the reader to interactively explore a generalised overview and hence also the visualisation. The reader should be able to focus on a part of the overview and dig into the underlying literature. The arrangement and the sequence of the contents in the literature need to be visualised for facilitating comparative analysis. The arrangement of the comparing literatures in the visualisation also plays an important role in the comparison tasks. Moreover our visual analysis tool should address the arbitrary changes to the visualised contents like resizing of the visualisation window and other screen changes.

R5: Interactivity
The interactive methods allow readers to dig into the literature and learn more about the topics. It should allow user to navigate through literature effortlessly.
1.	Quick text previews
The reader should be able to inspect certain areas of the visualisation in detail without losing overview of the literatures. The reader should be able to quickly go through the underlying text in the literature containing interesting topics or phrases. The various occurrences of the topics in the literatures should be informed to the reader so that he could decide which part of the literature they want to explore in details.

2.	Details on Demand
The reader should be given the power to interactively drill down from a generalised overview to shortened text paragraphs and up to the full text of the literature. The reader should be able to select and focus on the sections from literature for comparison. The selection should not be restricted only to the sections but it should also provide possibility to select topics and storylines for detailed exploration. The visualisation should adapt to the reader's desired specialisations without preserving the generalised overview.

3.	Search for keywords to retrieve details
The visualization based on automatically identified and ranked storylines, might miss out the information that is of reader's interest. Such cases must be addressed by providing functionality to search for keywords to retrieve their detailed context in the literatures. The visualization should accommodate the comparative contextual information based on the searched keywords.
