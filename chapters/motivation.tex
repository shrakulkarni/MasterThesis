\chapter{Motivation}
\label{motivation}

We read books for amusement as well as to gain knowledge. Books are the traditional and important source of information. There often occur situations where we need to compare two or more books. For comparing book readers quickly run through chapters and pages of books to get the overview. Not only professional like students, linguist or historian but also other hobby readers often compare books in order to decide which book is more interesting to read. They have different purposes for such comparison. The hobby readers look for the general topics and themes of the books. Professional readers, who are already aware of the themes of books, are interested in the description of specific topics. In the process, they might overlook important information. This problem motivates us to facilitate the readers with an interactive visualization tool for comparative book analysis.

We aim to provide comparative book analysis with the help of an interactive visualization and thus to ease the comparative book analysis by providing reader contextual overview of similarities and differences of the books. The overview should be able to show clear differentiation between similar and different themes of the two books. The visualization should provide generalized overview of the various themes in the books which in turn should help readers to comprehend its context. Depending on the readers’ requirements, visualization should be able to adapt and focus on the readers’ query. The visualizations should precisely express changes in themes and structures of the books. It should be able to visualize the tendencies of the themes in the books. The structural and grammatical differences like sentence formats will aid the context awareness of the visualization. We aim to provide interactive visualization which allows user-guided exploration of the themes in books.